% Demasiados detalles de Redes de Petri

%\begin{frame}{Representación formal}{Redes de Petri}
%    \begin{itemize}
%      \setlength\itemsep{0.3cm}
%      \item<2-> Existen diferentes representaciones formales utilizados en la minería de procesos.
%                En particular, usaremos las \textbf{redes de Petri} como modelo de representación.
%      \item<3-> Las redes de Petri fueron introducidas en el año 1962 por el matemático Carl Adam Petri
%                para representar sistemas dinámicos con eventos concurrentes.
%      \item<4-> Conforman un lenguaje gráfico y matemático con una semántica formal.
%      \item<5-> Desde su creación han sido utilizadas, entre otros usos, para representación,
%                análisis, verificación y simulación de sistemas de eventos discretos con 
%                comportamiento dinámico.
%    \end{itemize}
%\end{frame}
%
%\begin{frame}{Redes de Petri}{Definición formal}
%  \begin{itemize}
%    \item<2-> Formadas por dos componentes:
%      \begin{itemize}
%        \setlength\itemsep{0.2cm}
%          \item<3-> Un grafo bipartito cuyos nodos se separan en los conjuntos disjuntos llamados
%                    \textit{places} y \textit{transiciones} representa la red propiamente dicha.
%          \item<4-> Un conjunto de fichas asignadas a los places de la red, denominado \textit{marking},
%                    el cual es utilizado para simular el comportamiento dinámico y concurrente del sistema.
%      \end{itemize}
%  \end{itemize}
%
%  \pause[5]
%
%%  \begin{tcolorbox}[colback=gray!5!white,colframe=gray!50!black,
%%  colbacktitle=gray!75!black,title=Red de Petri]
%  \begin{block}{Redes de Petri}
%    Formalmente, una red de Petri es una 4-upla $(P,T,F,M_0)$ donde:
%     \begin{itemize}
%        \item<6->{$P$ representa el conjunto finito de places.}
%        \item<7->{$T$ representa el conjunto finito de transiciones.}
%        \item<8->{La función \mbox{$F:(P \times T) \cup (T \times P)  \to \nat$} asigna el peso a los diferentes arcos.}
%        \item<9->{Un marking inicial dado por la función \mbox{$M_0:P \to \nat$}.}
%     \end{itemize}
%  %\end{tcolorbox}
%  \end{block}
%\end{frame}
%
%\begin{frame}{Redes de Petri}{Evolución}
%    \begin{itemize}
%      \setlength\itemsep{0.2cm}
%      \item<2-> El dinamismo de una red viene dado por los markings, los cuales se van ``moviendo''
%                de un place a otro tras la ejecución de las transiciones \emph{habilitadas}.
%      \item<4-> Una transición $t$ se encuentra habilitada, en un marking $M$ si 
%                contiene al menos tantas fichas en cada place que incide en $t$
%                como marcan los arcos que los conectan:
%                \bnnequation
%                  \mbox{$\forall p \in P:~ M(p) \ge F(p,t) $}.
%                \ennequation
%      \item<5-> Ejecutar una transición $t$ sobre un place $p \in P$ en un cierto marking $M$ genera 
%                un nuevo marking $M'$: \firing{M}{t}{M'}. 
%      %\item<2-> Dada una red de Petri $N$, se llama $\Language(N)$ al conjunto de secuencias de transiciones ejecutables sobre $N$.
%      %\item<3-> Por su parte, al conjunto de markings alcanzable partiendo desde el marking inicial $M_0$,
%      %          llamado \emph{conjunto alcanzable} de $N$, se lo nota $\rs(N)$.
%  \end{itemize}
%\end{frame}

%\begin{frame}{Parikh vector}{Definición}
%  \begin{itemize}
%    \setlength\itemsep{0.2cm}
%    \item<2->
%Dada una traza $\sigma=\sigma_1\cdot\sigma_2\cdot\ldots\cdot\sigma_k$ sobre un alfabeto
%$T=\{t_1,t_2,\dots,t_n\}$, se utiliza $|\sigma|_{t_i}$ para representar la
%cantidad de ocurrencias de $t_i$ en $\sigma$.
%    \item<3->
%Luego, el \emph{vector Parikh} de $\sigma$ se define como un vector
%de longitud $n$ donde la $i$-esima componente toma como valor la cantidad de ocurrencias
%de la actividad $t_i$.
%  \end{itemize}
%
%   \pause[4]
%   \begin{example}
%      Si consideramos las trazas $\sigma_1=t_1 \cdot t_2 \cdot t_1 \cdot t_2 \cdot t_1 \cdot t_1 \cdot t_3$ y $\sigma_2=t_4 \cdot t_4$,
%      sobre el alfabeto $T=\{t_1,t_2,t_3,t_4\}$ los vectores de Parikh de cada traza vienen dados por las tuplas $(4,1,1,0)$ y $(0,0,0,2)$ respectivamente.
%   \end{example}
%\end{frame}
%
%\begin{frame}{Parikh vector}{Definición formal}
%  \begin{block}{Vector Parikh}
%    Sea $T=\{t_1,\ldots,t_n\}$ un alfabeto de actividades,
%    el vector Parikh de una secuencia de eventos sobre $T$ se define como
%    la función
%
%    \bequation
%        \begin{array}{lll}
%            \widehat{\ }:& T^*&\rightarrow \nat^n\\
%            \;&\sigma &\mapsto \widehat{\sigma}=(|\sigma|_{t_1},\dots, |\sigma|_{t_n} ).
%        \end{array}
%    \eequation
%  \end{block}
%  \pause[2]
%  \begin{block}{Vectores Parikh de un log de eventos}
%    Dado un log $\pmlog$, el conjunto de
%    vectores Parikh de $\pmlog$ se define como
%
%    \bequation
%        \parikh{\pmlog}=\{ \widehat\sigma ~|~ \sigma \in \pmlog \}.
%    \eequation
%  \end{block}
%  \begin{columns}
%    \column{0.3\textwidth}
%        \pause[4]
%        \centering
%        \includegraphics[width=1.2\linewidth]{img/homer-simpson-cooking.jpg}
%    \column{0.7\textwidth}
%        \pause[3]
%        Entonces, teníamos un log de eventos y ahora podemos convertirlas en tuplas de enteros...
%  \end{columns}
%\end{frame}

%\begin{frame}{Simplificación}{Simplificando la simplificación}
%  \begin{itemize}
%    \setlength\itemsep{0.4cm}
%    \item Para realizar el proceso de simplificación se utiliza \textit{SMT-Solver}.
%    \item SMT (\textit{Satisfability modulo theories}) es una teoría que permite
%        definir un sistema de restricciones y encontrar una solución.
%  \end{itemize}
%
%\end{frame}
